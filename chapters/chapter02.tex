\chapter{轨迹规划的理论基础}
在进行轨迹规划之前,首先要明确轨迹规划的对象,已知无人机具有12维的状态参数,在做轨迹规划的过程中不可能对全维度空间进行规划,但是可以利用微分平坦理论证明只需对轨迹状态进行规划就可以达到控制无人机的目的,下面将对此进行证明。此外本章还介绍了后面章节会用到的贝塞尔曲线和优化问题,主要说明了曲线的基本特性和二次规划问题及其主要求解方式。

\section{无人机系统的微分平坦性}
微分平坦特性是系统本身的一种结构属性,一般地,对于一个控制系统来说,如果存在一组输出可以使得控制系统中的所有状态变量和控制输入变量都可以表示为这组输出及其有限阶微分的函数表达式,则认为此系统时微分平坦的。对于如式 \ref{equ_1} 所示的非线性系统:
\setlength\abovedisplayskip{0.15cm}
\setlength\belowdisplayskip{0.15cm}%公式
\begin{equation}
    \left\{ \begin{array}{l}
    \dot x = f(x,u),x \in {R^n},u \in {R^m}\\
    y = g(x),y \in {R^m}
    \end{array} 
    \right.
    \label{equ_1}
\end{equation}
其中$u$为输入,$y$为输出,$x$为系统状态变量,当系统存在输出${O = h(x,u,\dot u, \cdots ,{u^{(n - 1)}})}$使得:
\begin{equation}
    \begin{array}{l}
        x = x(O,\dot O, \cdots ,{O^{(n - 1)}})\\
        u = u(O,\dot O, \cdots ,{O^{(n - 1)}})
        \end{array}
        \label{equ_2}
\end{equation}
则称非线性系统是微分平坦的,此系统的微分平坦输出为${O}$。已知微分平坦输出${O}$的期望轨迹${O_d}$,就可以依据微分平坦的定义的到系统所有变量的期望表达式。

依据上述理论可以知道系统状态、输入和平坦输出与之间的关系是一一对应的,也就是所微分平坦系统的状态特性可由平坦输出唯一决定。同时,具有微分平坦特性的系统可以将状态变量$x$和控制输入$u$映射到平坦的输出空间中,而输出空间的维数总是低于状态空间$x$的维数,因此微分平坦理论常常用于轨迹规划问题。应用该理论可以将原本复杂的高维空间轨迹规划问题转化到低维的平坦空间处理,在该空间获得求解结果后通过一定的方式映射得到原状态空间的轨迹。此外,微分平坦特性还有个优点,它并不依赖于坐标系的选取,这同样也利于简化问题,便于轨迹规划。如果四旋翼无人机控制系统满足上述的微分平坦性,那么四旋翼无人机系统的输入量和状态栏都可以用此平坦输出及其微分表示。因此,下面将在理论方面对此进行论证\ucite{谈冰然2020基于空气动力学补偿的四旋翼无人机飞行控制与轨迹规划}。


\section{贝塞尔曲线}

贝塞尔曲线是一种以伯恩思坦多项式为基础的样条曲线,$n$阶贝塞尔曲线的多项式由$n+1$个线性无关的伯恩思坦基组成,并由$n+1$个控制点控制\ucite{sederberg2012computer}。


\endinput